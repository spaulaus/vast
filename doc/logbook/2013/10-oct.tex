\chapter{October}
\label{ch:oct}
\section{10-28-2013}
\label{sec:10282013}
\subsection{Intro}
This is the first entry into the logbook. For more information about the work 
that has been done thus far one should check out the October 2013 section of 
the \iso{Cu}{77} logbook. 

\subsection{Error Propagation}
We need to get the error propagation going properly for the various 
calculations that are going down. This error will need to be carried around 
for all of the various parameters that have it. Either from the literature or 
from the experimental data themselves. There are a few parameters for which 
we will ignore the error: the speed of light and the rest mass of the neutron. 
These two are experimentally determined pretty well. 

\section{10-29-2013}
\label{sec:10292013}
We are continuing with the error propagation stuff, we will be working on the 
sensitivity limit stuff after this is completed. 

\subsection{Future plans}
We need to get an xml setup file prepared, and integrate the pugixml codes. 
This will make setting all of the various file names/paths/etc. easier. It 
will also allow one to make changes to the setups without having to recompile 
the code. I would also like to get thing setup so that the ROOT libraries are 
only linked if they have to be. This is a quick and easy switch in the Makefile.

\subsection{Errors}
The error on $\mu$ and yield are provided by the fit. I need to include 
errors on the $\sigma$ from that fit as well, which need to come from the 
fit to the parameterization of $\sigma$. We are gonig to make a convention that 
the errors are all called $\rho$ for this, there are too many $\sigma$'s 
floating around to keep track of things properly.

We need to start tracking the error in the calculations from the highest level 
and work our way down. Maybe an ErrorCalculator class will be helpful for 
handling these things. I can just pass it the variables that need to be 
calculated. The multiplication by any constants can happen in the class itself.
The error class just needs to know the final variable the power if 
using multiplication, and the general process?

\subsection{Summary of Errors - From Bevington}
This is a summary of the information contained at the end of Chapter 2 of 
\cite{bevington2003}.

\noindent \textbf{Covariance:} 
\begin{equation}
\sigma_{uv}^2 = \langle(u-\bar{u})(v-\bar{v})\rangle
\end{equation}

\noindent \textbf{Propagation of errors: Assume $x=f(u,v)$:}
\begin{equation}
\sigma^2_x = \sigma_u^2\left(\pdiff{x}{u}{}\right)^2 +
   \sigma_v^2\left(\pdiff{x}{v}{}\right)^2 + 
   2\sigma_{uv}^2\left(\pdiff{x}{u}{}\pdiff{x}{v}{}\right)
\end{equation}
For $u$ and $v$ uncorrelated, $\sigma_{uv}^2 = 0$.

\noindent \textbf{Specific Formulas:}
\begin{table}[h]
  \begin{center}
    \begin{tabular}{l c}
      $x=au+bv$ & $\sigma_x^2 = a^2\sigma_u^2+b^2\sigma_v^2 + 2ab\sigma_{uv}^2$ \\
      $x=auv$ & $\frac{\sigma_x^2}{x^2} = \frac{\sigma_u^2}{u^2} + 
         \frac{\sigma_v^2}{v^2}+2\frac{\sigma_{uv}^2}{uv} $ \\
      $x=\frac{au}{v}$ & $\frac{\sigma_x^2}{x^2} = \frac{\sigma_u^2}{u^2} + 
         \frac{\sigma_v^2}{v^2}-2\frac{\sigma_{uv}^2}{uv} $ \\
      $x=au^b$ & $\frac{\sigma_x}{x} = b\frac{\sigma_u}{u}$ \\
      $x=ae^{bu}$ & $\frac{\sigma_x}{x} = b\sigma_u$ \\
      $x=a^{bu}$ & $\frac{\sigma_x}{x} = (b \mathrm{ln} a) \sigma_u$ \\
      $x=a\mathrm{ln}(bu)$ & $\sigma_x = ab\frac{\sigma_u}{u}$ \\
      $x=a \cos(bu)$ & $\sigma_x = -\sigma_uab\sin(bu)$ \\
      $x=a \sin(bu)$ & $\sigma_x = \sigma_uab\cos(bu)$ \\
    \end{tabular}
  \end{center}
  \caption{Specific formulas for error propagation}
  \label{tbl:errFormulas}
\end{table}

Now I no longer have to look all this stuff up. 

\section{10-31-2013 - Happy Halloween!!}
\label{sec:10312013}
\subsection{Analysis Workflow}
I am going to detail here the process from start to finish of analyzing the 
VANDLE data. 
\begin{enumerate}
\item Calibrate the gamma spectra for the scan code, using files from the 
  calibration sources. This calibration may need to be done using gamma 
  rays from the ldfs that one wantts to analyze. 
\item Calculate the effiency ($\epsilon_\gamma(E_\gamma)$) of the clovers. 
  This needs to be done with the calibrated source data. Currently this 
  information resides in a spreadsheet titled ``efficiency.ods ``.
\item Use the gamma singles spectrum to determine the most intense (clean) 
  $\gamma$ for which the abolute branching ratio (ABR$_\gamma$) is known. 
  Take care to find a gamma line that does not have a component of direct 
  implantation in the beam. Fit this gamma line to obtain the intensity 
  of the $\gamma$ in singles, A$_\gamma$.
\item Calculate the absolute number of $\beta$ decays using the information 
  from the previous points.
  \begin{equation}
    N_{dky} = \frac{A_\gamma}{\epsilon_\gamma(E_\gamma)ABR_\gamma}
    \label{eqn:numDecays}
  \end{equation}
\item Calculate the efficiency of the $\beta$ detectors, $\epsilon_\beta$.
  To determine this take the ratio of $\gamma$s from the $\beta$-n 
  daughter in the $\gamma$ singles and $\beta$-gated spectra. This number 
  needs to be determined a little bit more preciesly, but for now this 
  is going to be good enough. One may consider taking an average if more 
  than one line from the beta delayed neutron daughter is found.
  \begin{equation}
    \epsilon_\beta(E_\gamma) = \frac{A_{gated}}{A_{singles}}
    \label{eqn:betaEff}
  \end{equation}
\item Calculate or take from simulation the solid angle coverage of 
  VANDLE. We are currently using a value of 0.0061 for a single bar, 
  which comes from Sergey's simulation. A calculation of this number 
  provides a value of 0.004727. This parameter will be treated as a constant.
\item Calculate the VANDLE efficiency, $\epsilon_{v}(E_n)$. These values can 
  either be taken from the Ohio or Cf data, or determined as a function of the 
  banana curve using GEANT4. Currently, we are using the form of the 
  efficiency from GEANT4 in the calculations, and will be treating the error of 
  the efficiency as constant for now. We should find the relative error in the 
  simulation from the comparison with the Ohio data, and then figure out the 
  percentage that its off as a function of energy.
\item Determine the reoslution of the start detector. This should be 
  parameterized in terms of the ToF. The parameterization of this variable 
  will have error associated with it, but the error shall be ignored for now. 
  The parameterization currently in use comes from a spreadsheet by Miguel and 
  is titled ``VANDLE\_bar\_QDC\_calibratio\_test.ods''.
\item Convolute the neutron response curve from GEANT4 with a Gaussian that has 
  a resolution given from the previous point.
\item Repeat the previous point for N response curves. This will allow one to 
  parameterize $\alpha$, n, and $\sigma_{ToF}$ as a function of $\mu$.
\item Perform a fit to the $\alpha$,n,$\sigma_{ToF}$ vs. $\mu$ curves to 
  parameterize them in terms of $\mu$.
\item Perform the deconvolution of the neutron spectrum using a CB PDF that 
  has $\alpha$,n,$\sigma_{ToF}$ given by the previous point. The only free 
  parameters in this fit will be $\mu_i$ and y$_i$. The errors fro these two 
  parameters will be taken from the fit itself. Currently, the error on 
  the parameterization steps is ignored.
\item Convert the $\mu_i$s into E$_i$:
  \begin{equation}
    E_i(\mu_i) = \frac{1}{2}m_n \left(\frac{d}{\mu_ic}\right)^2
    \label{eqn:kinE}
  \end{equation}
\item Calculate the full neutron yield, y$_i'$, by integrating the CB from 
  [0,$\infty$]. In practice, the calculation is performed by integrating 
  the CB from [fitLow, fitHigh] this provides the normalization of the integral,
  and then from [fitHigh,$\infty$]. The error is calculated assuming that the 
  precision of the numeric integral is much smaller than the error on the yield, 
  which is going to be the case for all of the cases studied.
  \begin{equation}
    y_i' = \frac{y_i}{\int_{fitLow}^{fitHigh}CB}\int_{fitHigh}^{\infty}CB + y_i
    \label{eqn:fullYld}
  \end{equation}
\item Calculate the branching ratio of the neutron lines, BR$_i$. You only need 
  to include the gamma efficiency if the BR$_i$ is calculated for the gamma 
  gated neutron spectrum.
  \begin{equation}
    BR_i = \frac{y_i'}{N_{dky} \Omega \epsilon_\beta \epsilon_\gamma \epsilon_v}
    \label{eqn:brCalc}
  \end{equation}
\item Calculate the log(ft)$_i$. The error on the f is going to be swamped by the 
  error on the t$_{1/2}$ and the BR$_i$. 
  \begin{equation}
    log(ft)_i = \log\left(\frac{f t_{1/2}}{BR_i}\right)
    \label{eqn:logft}
  \end{equation}
\item Calculate the B(GT)$_i$ for each of the neutron lines:
  \begin{equation}
    B(GT)_i = \frac{D*BR_i}{(g_a/g_v)^2f t_{1/2}}
    \label{eqn:bgt}
  \end{equation}
\item Calculate the neutron density by spreading the BR$_i$ using a Gaussian 
  distribution. The mean of the distribution will be the $\mu_i$, the variance 
  will be given by 
  \begin{equation}
    \sigma_{Ei} = E_i(\mu_i - 0.25\sigma_{i}') - E_i(\mu_i + 0.25\sigma_{i}').
    \label{eqn:bgtWidth}
  \end{equation}
  The amplitude will be given by the amplitude of the BR$_i$, which is 
  scaled according to the sampling resolution so that the total Gaussian 
  area is independent of the step size.
\item Convert the neutron density spectrum into B(GT). The error of the point 
  heights will be given by the equations in the next section. 
\item Repeat steps 12 - 19 with the gamma gated neutron spectra if the statistics
  allow for it.
\end{enumerate}

\subsection{Summary of Parameters with Error}
\begin{enumerate}
\item The error in the beta efficiency, $\epsilon_\beta$:
  \begin{equation}
    \left(\frac{\sigma_{\epsilon\beta}}{\epsilon_\beta}\right)^2 = 
    \left(\frac{\sigma_{Agated}}{A_{gated}}\right)^2 + 
    \left(\frac{\sigma_{Asngl}}{A_{sngl}}\right)^2
    \label{eqn:errBeta}
  \end{equation}
\item The error in the location of the gamma energies, E$_\gamma$ is given 
  simply by statistics. The 3.3 is the 99.9\% level of confidence for a 
  normal distribution.
  \begin{equation}
    \sigma_{\gamma}^2 = \frac{3.3 * FWHM}{2\sqrt{2N_{binsFit}\ln{2}}}
    \label{eqn:errGammaEn}
  \end{equation}
\item The error on the gamma efficiency, $\epsilon_\gamma$ will be approximated 
  using the MC techniques that Michael taught me. The distribution is a 
  log-normal distribution and the variance is described as
  \begin{equation}
    VAR = (e^{\sigma^2} - 1) e^{2\mu+\sigma^2}
    \label{eqn:logNormalVariance}
  \end{equation}
  Where the $\mu$ is the sample average, and the sample variance, $\sigma^2$,
  is defined as
    \begin{equation}
    \sigma_y^2 = \frac{1}{n}\sum_{i=1}^n(y_i-\mu)^2
    \label{eqn:sampleVariance}
  \end{equation}
\item Error on the number of decays, N$_{dky}$
  \begin{equation}
    \left(\frac{\sigma_{dky}}{N_{dky}}\right)^2 = 
    \left(\frac{\sigma_{A\gamma}}{A_\gamma}\right)^2 + 
    \left(\frac{\sigma_{\epsilon\gamma}}{\epsilon_\gamma}\right)^2 +
    \left(\frac{\sigma_{ABR\gamma}}{ABR_\gamma}\right)^2
    \label{eqn:errNdky}
  \end{equation}
\item Error on the neutron energies, E$_i$. This equation takes into account 
  the error from interaction point, and the timing resolutions from the 
  $\beta$, see \cite{kornilov2009}.
  \begin{equation}
    \sigma_{Ei} = E_i(\mu_i - 0.25\sigma'_i) - E_i(\mu_i + 0.25\sigma'_i)
    \label{eqn:errEnergy}
  \end{equation}
\item Error on the integrated neutron yields, y$_i'$
  \begin{equation}
    \sigma_{y'}^2 = \frac{\sigma_{y} \int_{fitHigh}^{\infty}CB}{\int_{fitLow}^{fitHigh}CB}
    + \sigma_y^2
    \label{eqn:errIntYield}
  \end{equation}
\item Error on the integrated neutron yields adjusted for efficiency, y$_i''$
  \begin{equation}
    \left(\frac{\sigma_{yi''}}{y_i''}\right)^2 = 
    \left(\frac{\sigma_{\epsilon V}}{\epsilon_V}\right)^2 + 
    \left(\frac{\sigma_{yi'}}{y_i'}\right)^2
    \label{eqn:errIntYieldEff}
  \end{equation}
\item Error on the Branching Ratio, BR$_i$. The error for $\epsilon_V$ is not include 
  because it is included as part of the calculation for the integrated neutron yield.
  \begin{equation}
    \left(\frac{\sigma_{BRi}}{BR_i}\right)^2 = 
    \left(\frac{\sigma_{yi''}}{y_i''}\right)^2 + 
    \left(\frac{\sigma_{dky}}{N_{dky}}\right)^2 +
    \left(\frac{\sigma_{\epsilon\gamma}}{\epsilon_\gamma}\right)^2 +
    \left(\frac{\sigma_{\epsilon\beta}}{\epsilon_\beta}\right)^2
    \label{eqn:errBR}
  \end{equation}
\item Error on the B(GT), this is only the error on the amplitude of the points, 
  not the error on the width.
  \begin{equation}
    \left(\frac{\sigma_{BGT}}{BGT}\right)^2 = 
    \left(\frac{\sigma_{BR}}{BR}\right)^2 + 
    \left(\frac{\sigma_{t1/2}}{t_{1/2}}\right)^2
    \label{eqn:errBGT}
  \end{equation}
\item Error on the log(ft)
  \begin{equation}
    \sigma_{lotft}^2 = \left(\frac{\sigma_{BR}}{BR\ln(10)}\right)^2 +
    \left(\frac{\sigma_{t1/2}}{t_{1/2}\ln(10)}\right)^2
    \label{eqn:errLogFT}
  \end{equation}
\end{enumerate}
