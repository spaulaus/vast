\chapter{October}
\workday{10-04-2014}
\subsection{Change Tack}
I am going to change my testing files from the \iso{Cu}{77} files to the 
\iso{Ga}{84} files. This way we can have a good comparison with Miguel's work
and confirm that there is nothing wrong with my codes. There is still this 
huge discrepancy with the \iso{Cu}{77,78} data that needs to be worked out and
I'm at the point where I really need to just confirm that my codes are doing 
what we think that they are doing. So let's get rolling.

\subsection{Update Eff Function}
We are going to need to make sure that we have the proper efficiency function 
from the simulation results here. I have been lazy with the efficiency 
calculator and have them all copy/pasted into the codes, so we should have it 
in there.

\subsection{Branching Ratio Calculation}
After revamping the EffCalculator, the B(GT) calculation has broken like a 
champ. Looks like I broke the Ge efficiency calculation. There was a unit error. 
I was trying to go from MeV to keV by dividing by 1000, this is dumb.

\subsection{P$_n$ for \iso{Ga}{84} - MMF}
I have got all the major errors that I can find fixed. This leaves us with the 
more complicated stuff. The P$_n$ for the \iso{Ga}{84} is being reported as 
40\%. This is significantly smaller than what Miguel reports, and what I have
gotten in the past. 

\textbf{The setup for this number is } with the ToF data taken using Miguel's 
banana, and the efficiency curve from Sergey using Miguel's banana. I have 
listed the current values for things in Table \ref{tbl:ga84-mmf-vast}.

\begin{table}[htbp]
  \caption{The table containing the potentially correct results from my code 
    for the \iso{Ga}{84} data. The data use ToFs from Miguel's banana and 
    Sergey's efficiency from that banana.}
  \label{tbl:ga84-mmf-vast}
  \begin{center}
    \begin{tabular}{|c|c|c|}
      \hline
      Paramter & Value & Error\\\hline
      Integrated Yield & 2.662e5 & --\\\hline
      Number of Decays & 6.7503e5 & --\\\hline
      Pn & 0.39435 & 0.95527\\\hline
      Raw Number of Neutrons & 1011 & --\\\hline
      Raw Integrated Number of Neutrons & 1199.2 & --\\\hline
    \end{tabular}
  \end{center}
\end{table}

I am double checking that I haven't screwed anything obvious up in my 
calculations. I need to really make sure that things are clean and easy to 
follow in terms of what they are doing. 

\textbf{UPDATE: } I have just gotten an email from Miguel saying that this number
that I am getting is consistent with what he is getting! This is good news. I 
am going to call that a win for now, and will talk with him more tomorrow.

\subsection{P$_n$ for \iso{Ga}{84} - SVP}
I have done the same calculations as above using my own banana, and have come up
with a number that's about 10\% lower than the numbers using information that 
is all taken from Miguel's banana/eff. The results are summarized in Table 
\ref{tbl:ga84-svp-vast-wrong}
\begin{table}[htbp]
  \caption{The table containing the incorrect results from my code 
    for the \iso{Ga}{84} data. The data use ToFs from my banana and 
    Sergey's efficiency from that banana.}
  \label{tbl:ga84-svp-vast-wrong}
  \begin{center}
    \begin{tabular}{|c|c|c|}
      \hline
      Paramter & Value & Error\\\hline
      Integrated Yield & 2.0414e5 & --\\\hline
      Number of Decays & 6.7503e5 & --\\\hline
      Pn & 0.30241 & 0.4472\\\hline
      Raw Number of Neutrons & 721.97 & --\\\hline
      Raw Integrated Number of Neutrons & 856.34 & --\\\hline
    \end{tabular}
  \end{center}
\end{table}

This result seems to support the idea that there is something iffy with the 
efficiencies from my data. It's also possible that my banana may not be 
working the greatest? 

From the results between the two different versions I see that the peaks 
are not quite in the same spots. This could be throwing things a little 
off, but I do not expect it to be by 10\%. I am rerunning the data with 
the efficiency curve taken from mmfBan instead of svpBan4. I would like to 
see if the results change significantly or not. 

The results do change a little bit. The P$_n$ is now 0.28411 coming back down 
from the 30\% that I had previously. This supports the idea that the efficiency
is indeed recovering for the cuts my banana is making, but it doesn't seem to 
be quite enough. We will go through the various different efficiency curves to 
see what is going on here. The ``rolling'' efficiency curve gets me back up to 
approximately 34\%. It seems that none of the efficiency curves really 
compensate properly for the cuts that my banana gives. I should double check 
that values that I am providing to Sergey are indeed the proper ones. 

\subsection{Documentation}
I need to clean up some of the documentation for the program. Most importantly 
how to make sure that some of the relevant information gets into the Doxygen 
generated stuff. 

\workday{10-05-2014}
\subsection{Error Calculations}
Some of the error calculations are fubar. We will need to delve into those as 
soon as the efficiency issue is resolved. 

\subsection{Continuing to check efficiencies}
I am looking at the different efficiencies that we have available, and none of 
them seem to do very well with my banana. The VANDLE efficiency curve is the 
last and only cluprit here. 
